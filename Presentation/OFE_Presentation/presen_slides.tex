% Options for packages loaded elsewhere
\PassOptionsToPackage{unicode}{hyperref}
\PassOptionsToPackage{hyphens}{url}
%
\documentclass[
  12pt,
  ignorenonframetext,
  aspectratio=169,
]{beamer}
\title{Machine Learning Methods for Site-specific Input Management}
\author{Shunkei Kakimoto\(^1\), Taro Mieno\(^1\) and Takashi S. T.
Tanaka\(^2\)}
\date{September 1st 2021}
\institute{\(^1\)Agricultural Economics, University of Nebraska Lincoln
\(^2\)Applied Biological Sciences, Gifu University}

\usepackage{pgfpages}
\setbeamertemplate{caption}[numbered]
\setbeamertemplate{caption label separator}{: }
\setbeamercolor{caption name}{fg=normal text.fg}
\beamertemplatenavigationsymbolsempty
% Prevent slide breaks in the middle of a paragraph
\widowpenalties 1 10000
\raggedbottom
\setbeamertemplate{part page}{
  \centering
  \begin{beamercolorbox}[sep=16pt,center]{part title}
    \usebeamerfont{part title}\insertpart\par
  \end{beamercolorbox}
}
\setbeamertemplate{section page}{
  \centering
  \begin{beamercolorbox}[sep=12pt,center]{part title}
    \usebeamerfont{section title}\insertsection\par
  \end{beamercolorbox}
}
\setbeamertemplate{subsection page}{
  \centering
  \begin{beamercolorbox}[sep=8pt,center]{part title}
    \usebeamerfont{subsection title}\insertsubsection\par
  \end{beamercolorbox}
}
\AtBeginPart{
  \frame{\partpage}
}
\AtBeginSection{
  \ifbibliography
  \else
    \frame{\sectionpage}
  \fi
}
\AtBeginSubsection{
  \frame{\subsectionpage}
}
\usepackage{amsmath,amssymb}
\usepackage{lmodern}
\usepackage{iftex}
\ifPDFTeX
  \usepackage[T1]{fontenc}
  \usepackage[utf8]{inputenc}
  \usepackage{textcomp} % provide euro and other symbols
\else % if luatex or xetex
  \ifXeTeX
    \usepackage{mathspec}
  \else
    \usepackage{unicode-math}
  \fi
  \defaultfontfeatures{Scale=MatchLowercase}
  \defaultfontfeatures[\rmfamily]{Ligatures=TeX,Scale=1}
\fi
\usetheme[]{Madrid}
\usecolortheme{whale}
\usefonttheme{professionalfonts}
% Use upquote if available, for straight quotes in verbatim environments
\IfFileExists{upquote.sty}{\usepackage{upquote}}{}
\IfFileExists{microtype.sty}{% use microtype if available
  \usepackage[]{microtype}
  \UseMicrotypeSet[protrusion]{basicmath} % disable protrusion for tt fonts
}{}
\makeatletter
\@ifundefined{KOMAClassName}{% if non-KOMA class
  \IfFileExists{parskip.sty}{%
    \usepackage{parskip}
  }{% else
    \setlength{\parindent}{0pt}
    \setlength{\parskip}{6pt plus 2pt minus 1pt}}
}{% if KOMA class
  \KOMAoptions{parskip=half}}
\makeatother
\usepackage{xcolor}
\IfFileExists{xurl.sty}{\usepackage{xurl}}{} % add URL line breaks if available
\IfFileExists{bookmark.sty}{\usepackage{bookmark}}{\usepackage{hyperref}}
\hypersetup{
  pdftitle={Machine Learning Methods for Site-specific Input Management},
  pdfauthor={Shunkei Kakimoto\^{}1, Taro Mieno\^{}1 and Takashi S. T. Tanaka\^{}2},
  hidelinks,
  pdfcreator={LaTeX via pandoc}}
\urlstyle{same} % disable monospaced font for URLs
\newif\ifbibliography
\setlength{\emergencystretch}{3em} % prevent overfull lines
\providecommand{\tightlist}{%
  \setlength{\itemsep}{0pt}\setlength{\parskip}{0pt}}
\setcounter{secnumdepth}{-\maxdimen} % remove section numbering
\newlength{\cslhangindent}
\setlength{\cslhangindent}{1.5em}
\newlength{\csllabelwidth}
\setlength{\csllabelwidth}{3em}
\newlength{\cslentryspacingunit} % times entry-spacing
\setlength{\cslentryspacingunit}{\parskip}
\newenvironment{CSLReferences}[2] % #1 hanging-ident, #2 entry spacing
 {% don't indent paragraphs
  \setlength{\parindent}{0pt}
  % turn on hanging indent if param 1 is 1
  \ifodd #1
  \let\oldpar\par
  \def\par{\hangindent=\cslhangindent\oldpar}
  \fi
  % set entry spacing
  \setlength{\parskip}{#2\cslentryspacingunit}
 }%
 {}
\usepackage{calc}
\newcommand{\CSLBlock}[1]{#1\hfill\break}
\newcommand{\CSLLeftMargin}[1]{\parbox[t]{\csllabelwidth}{#1}}
\newcommand{\CSLRightInline}[1]{\parbox[t]{\linewidth - \csllabelwidth}{#1}\break}
\newcommand{\CSLIndent}[1]{\hspace{\cslhangindent}#1}
\usepackage{fancyvrb}
\usepackage{graphicx}
\usepackage{booktabs}
\usepackage{longtable}
\usepackage{array}
\usepackage{multirow}
\usepackage{wrapfig}
\usepackage{float}
\usepackage{colortbl}
\usepackage{pdflscape}
\usepackage{tabu}
\usepackage{threeparttable}
\ifLuaTeX
  \usepackage{selnolig}  % disable illegal ligatures
\fi

\begin{document}
\frame{\titlepage}

\begin{frame}{Background}
\protect\hypertarget{background}{}
\begin{itemize}
\tightlist
\item
  The application of Machine Learning(ML) methods for site-specific
  economically optimal input rates (EOIR) (e.g., seed, fertilizer) have
  been getting more attention in recent years \newline

  \begin{itemize}
  \tightlist
  \item
    Barbosa et al. (2020) applied Convolutional Neural Network
  \item
    Krause et al. (2020) used Random Forest-based approaches
  \item
    Grant Gardner (2021), Wang et al. (2021), Coulibali, Cambouris, and
    Parent (2020), etc.
  \end{itemize}
\end{itemize}
\end{frame}

\begin{frame}{Research Gap}
\protect\hypertarget{research-gap}{}
\begin{itemize}
\tightlist
\item
  The conventional ML methods focus on predicting yield well rather than
  causal identification of input on yield
\item
  The past studies have used predictive ability of yield for validity of
  their models. \newline
\end{itemize}

\begin{block}{note}
\protect\hypertarget{note}{}
\begin{itemize}
\tightlist
\item
  EOIR estimation should be based on the change in yields associated
  with the change in input levels \emph{ceteris paribus}
\item
  Having good yield prediction capability does not necessarily mean it
  is also capable of estimating EOIR well
\end{itemize}
\end{block}
\end{frame}

\begin{frame}{New Trend of Causal Machine Learning Application}
\protect\hypertarget{new-trend-of-causal-machine-learning-application}{}
\begin{itemize}
\tightlist
\item
  Causal Machine Learning (CML) methods:

  \begin{itemize}
  \tightlist
  \item
    Unlike the conventional prediction-oriented ML methods, CML focuses
    on identifying causal impacts of an event (in our context, an
    increase or decrease in input rate for example) \newline
  \end{itemize}
\item
  \textbf{Causal Forest (CF) } (Wager and Athey 2018; Athey and Imbens
  2016):

  \begin{itemize}
  \tightlist
  \item
    CF estimates heterogeneous causal impacts of a treatment (a change
    in the input level) based on observed characteristics (e.g., organic
    matter)
  \end{itemize}
\end{itemize}
\end{frame}

\begin{frame}{Research Questions}
\protect\hypertarget{research-questions}{}
\begin{block}{}
\protect\hypertarget{section}{}
\begin{itemize}
\tightlist
\item
  In terms of estimating EOIR, how do CF-based methods (CF-stepwise and
  CF-base) compare to other prediction-oriented ML methods: Random
  Forest (RF), Boosted Random Forest (BRF), and Convolutional Neural
  Network (CNN)? \newline
\item
  Is the predictive ability of yield a good indicator of the performance
  of EOIR estimation?
\end{itemize}
\end{block}
\end{frame}

\begin{frame}{Methods}
\protect\hypertarget{methods}{}
\begin{itemize}
\tightlist
\item
  Conduct one thousand rounds of Monte Carlo simulations under four
  different production scenarios
\item
  Compare the performance in estimating economically optimal nitrogen
  rates (EONR) of CF-based methods (CF-base and CF-stepwise) to other
  methods: RF, BRF, and CNN
\item
  For RF, BRF, and CNN, contrast their EONR performances against their
  yield prediction performances
\end{itemize}
\end{frame}

\begin{frame}{Key Results: EONR estimation}
\protect\hypertarget{key-results-eonr-estimation}{}
\begin{center}\includegraphics[width=0.7\linewidth]{table_optN} \end{center}
\end{frame}

\begin{frame}{Key Results: EONR estimation and Yield Prediction}
\protect\hypertarget{key-results-eonr-estimation-and-yield-prediction}{}
\begin{center}\includegraphics[width=0.7\linewidth]{table_y} \end{center}
\end{frame}

\begin{frame}{Key Findings}
\protect\hypertarget{key-findings}{}
\begin{block}{}
\protect\hypertarget{section-1}{}
\begin{itemize}
\tightlist
\item
  The proposed CF-base method is capable of estimating site-specific
  EONR more accurately than other prediction-oriented ML methods
  \newline
\item
  The ability of predicting yield does not necessarily translate to good
  performance in estimating EONR.
\end{itemize}
\end{block}
\end{frame}

\begin{frame}[allowframebreaks]{References}
\protect\hypertarget{references}{}
\hypertarget{refs}{}
\begin{CSLReferences}{1}{0}
\leavevmode\vadjust pre{\hypertarget{ref-athey2016recursive}{}}%
Athey, Susan, and Guido Imbens. 2016. {``Recursive Partitioning for
Heterogeneous Causal Effects.''} \emph{Proceedings of the National
Academy of Sciences} 113 (27): 7353--60.

\leavevmode\vadjust pre{\hypertarget{ref-barbosa2020modeling}{}}%
Barbosa, Alexandre, Rodrigo Trevisan, Naira Hovakimyan, and Nicolas F
Martin. 2020. {``Modeling Yield Response to Crop Management Using
Convolutional Neural Networks.''} \emph{Computers and Electronics in
Agriculture} 170: 105197.

\leavevmode\vadjust pre{\hypertarget{ref-coulibali2020}{}}%
Coulibali, Zonlehoua, Athyna Nancy Cambouris, and Serge-Étienne Parent.
2020. {``Site-Specific Machine Learning Predictive Fertilization Models
for Potato Crops in Eastern Canada.''} \emph{PloS One} 15 (8): e0230888.

\leavevmode\vadjust pre{\hypertarget{ref-Gardner21}{}}%
Grant Gardner, David S Bullock, Taro Mieno. 2021. {``An Economic
Evaluation of Site-Specific Input Application r x Maps: Evaluation
Framework and Case Study.''} \emph{Precision Agriculture}.

\leavevmode\vadjust pre{\hypertarget{ref-krause2020random}{}}%
Krause, M, Savanna Crossman, Todd DuMond, Rodman Lott, Jason Swede,
Scott Arliss, Ron Robbins, Daniel Ochs, and Michael A Gore. 2020.
{``Random Forest Regression for Optimizing Variable Planting Rates for
Corn and Soybean Using Topographical and Soil Data.''}

\leavevmode\vadjust pre{\hypertarget{ref-Wager2018a}{}}%
Wager, Stefan, and Susan Athey. 2018. {``{Estimation and Inference of
Heterogeneous Treatment Effects using Random Forests}.''} \emph{Journal
of the American Statistical Association} 113 (523): 1228--42.
\url{https://doi.org/10.1080/01621459.2017.1319839}.

\leavevmode\vadjust pre{\hypertarget{ref-Wang2021}{}}%
Wang, Xinbing, Yuxin Miao, Rui Dong, Hainie Zha, Tingting Xia, Zhichao
Chen, Krzysztof Kusnierek, Guohua Mi, Hong Sun, and Minzan Li. 2021.
{``{Machine learning-based in-season nitrogen status diagnosis and
side-dress nitrogen recommendation for corn}.''} \emph{European Journal
of Agronomy} 123 (May 2020): 126193.
\url{https://doi.org/10.1016/j.eja.2020.126193}.

\end{CSLReferences}
\end{frame}

\end{document}
